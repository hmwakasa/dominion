% +++
% latex="texfot lualatex"
% +++
\documentclass{waku}
\begin{document}
\def\Expantion{帝国}\def\errata{2016}
\CardInfo{Engineer}{技術者}{CostA=\Debt{4}}{
	コスト\Coin{4}以下のカード 1 枚を獲得する。
	これを廃棄してもよい。そうした場合、追加で
	コスト\Coin{4}以下のカード 1 枚を獲得する。
}
\CardInfo{City Quarter}{市街}{CostA=\Debt{8}}{
	\Vanilla{action=2}
	あなたの手札を公開し、その中のアクションカード
	1 枚につき\PlusCard{1}。
}
\CardInfo{Overlord}{大君主}{CostA=\Debt{8},Type=アクション—命令}{
	サプライにあるコスト最大\Coin{5}までの
	命令でないアクションカードを、そこに置いたまま
	使用する。
}[2019私家]
\CardInfo{Royal Blacksmith}{王室の鍛冶屋}{CostA=\Debt{8}}{
	\Vanilla{card=5}
	あなたの手札を公開し、銅貨をすべて捨札にする。
}
\CardInfo{Encampment/Plunder}{陣地/鹵獲品}{Cost=2}{
	\hmafont
	この山札は上半分に陣地を 5 枚、下半分に鹵獲品
	を 5 枚置いてゲームを始める。この山札の一番
	上にあるカードだけが、獲得・購入可能。
}[2017HJ]
\CardInfo{Encampment}{陣地}{Cost=2,Copy=5}[このカードは陣地/鹵獲品の山札に由来する]{
	\Vanilla{card=2,action=2}
	あなたの手札から、金貨か鹵獲品を公開しても
	よい。そうしなかった場合、これを脇に置き、
	クリーンアップフェイズの開始時にサプライに
	戻す。
}
\CardInfo{Plunder}{鹵獲品}{Cost=5,Type=財宝,MC=2,Copy=5}[このカードは陣地/鹵獲品の山札に由来する]{
	\Vanilla(2){VP=1}
}
\CardInfo{Patrician/Emporium}{パトリキ/エンポリウム}{Cost=2}{
	\hmafont
	この山札は上半分にパトリキを 5 枚、下半分にエンポリウム
	を 5 枚置いてゲームを始める。この山札の一番
	上にあるカードだけが、獲得・購入可能。
}[2017HJ]
\CardInfo{Patrician}{パトリキ}{Cost=2,Copy=5}[このカードはパトリキ/エンポリウムの山札に由来する]{
	\Vanilla{card=1,action=1}
	あなたのデッキの一番上のカードを公開する。それが
	コスト\Coin{5}以上の場合、あなたの手札に加える。
}
\CardInfo{Emporium}{エンポリウム}{Cost=5,Copy=5}[このカードはパトリキ/エンポリウムの山札に由来する]{
	\Vanilla{card=1,action=1,coin=1}
	\hmbar
	これを獲得するとき、あなたの場に 5 枚以上の
	アクションカードがある場合、\PlusVP{2}。
}
\CardInfo{Settlers/Bustling Village}{開拓者/騒がしい村}{Cost=2}{
	\hmafont
	この山札は上半分に開拓者を 5 枚、下半分に騒がしい村
	を 5 枚置いてゲームを始める。この山札の一番
	上にあるカードだけが、獲得・購入可能。
}[2017HJ]
\CardInfo{Settlers}{開拓者}{Cost=2,Copy=5}[このカードは開拓者/騒がしい村の山札に由来する]{
	\Vanilla{card=1,action=1}
	あなたの捨札をすべて見る。その中の銅貨 1 枚を
	公開して手札に加えてもよい。
}
\CardInfo{Bustling Village}{騒がしい村}{Cost=5,Copy=5}[このカードは開拓者/騒がしい村の山札に由来する]{
	\Vanilla{card=1,action=3}
	あなたの捨札をすべて見る。その中の開拓者 1 枚を
	公開して手札に加えてもよい。
}
\CardInfo{Castles}{城}+{Cost=3,Type=勝利点—城,Color=勝利点}{
	\hmafont
	城の山札はコストが多いカードが高いカードが下に
	なるように並べ替えておく。
	2 人戦では各 1 枚のみ用いる。一番上のカードのみ購入・獲得可能。
}[2017HJ]
\CardInfo{Humble Castle}{粗末な城}
{Cost=3,Type=財宝—勝利点—城,Color=財宝,ColorB=勝利点,MC=1,Copy=2}[このカードは城の山札に由来する]{
	\Vanilla(1){}
	あなたの城 1 枚につき\VP[1]。
}
\CardInfo{Crumbling Castle}{崩れた城}
{Cost=4,Type=勝利点—城,Color=勝利点,Copy=1}[このカードは城の山札に由来する]{
	\Vanilla[1]{}
	これを獲得または廃棄するとき、\PlusVP{1}、
	そして銀貨 1 枚を獲得する。
}
\CardInfo{Small Castle}{小さい城}
{Cost=5,Type=アクション—勝利点—城,Color=アクション,ColorB=勝利点,Copy=2}[このカードは城の山札に由来する]{
	あなたの手札の城 1 枚またはこのカードを廃棄する。
	そうした場合、城 1 枚を獲得する。
	\hmbar
	\Vanilla[2]{}
}
\CardInfo{Haunted Castle}{幽霊城}
{Cost=6,Type=勝利点—城,Color=勝利点,Copy=1}[このカードは城の山札に由来する]{
	\Vanilla[2]{}
	\hmbar
	あなたがこれをあなたのターンに獲得するとき、
	金貨 1 枚を獲得し、手札が 5 枚以上の他のプレイヤー
	は全員、自分の手札 2 枚を自分のデッキの上に置く。
}
\CardInfo{Opulent Castle}{華やかな城}
{Cost=7,Type=アクション—勝利点—城,Color=アクション,ColorB=勝利点,Copy=2}[このカードは城の山札に由来する]{
	好きな枚数の勝利点カードを公開して捨札にする。
	捨札にしたカード 1 枚につき\PlusCoin{2}。
	\hmbar
	\Vanilla[3]{}
}[2021私家]
\CardInfo{Sprawling Castle}{広大な城}
{Cost=8,Type=勝利点—城,Color=勝利点,Copy=1}[このカードは城の山札に由来する]{
	\Vanilla[4]{}
	\hmbar
	これを獲得するとき、公領 1 枚か屋敷 3 枚を獲得する。
}
\CardInfo{Grand Castle}{壮大な城}
{Cost=9,Type=勝利点—城,Color=勝利点,Copy=1}[このカードは城の山札に由来する]{
	\Vanilla[5]{}
	\hmbar
	これを獲得するとき、あなたの手札を公開する。
	あなたの手札と場の勝利点カード 1 枚につき\PlusVP{1}。
}
\CardInfo{King's Castle}{王城}
{Cost=10,Type=勝利点—城,Color=勝利点,Copy=2}[このカードは城の山札に由来する]{
	あなたの城 1 枚につき\VP[2]。
}
\CardInfo{Catapult/Rock}{投石機/石}{Cost=2,Type=アクション—アタック}{
	\hmafont
	この山札は上半分に投石機を 5 枚、下半分に石
	を 5 枚置いてゲームを始める。この山札の一番
	上にあるカードだけが、獲得・購入可能。
}[2017HJ]
\CardInfo{Catapult}{投石機}{Cost=3,Type=アクション—アタック,Copy=5}[このカードは投石機/石の山札に由来する]{
	\Vanilla{coin=1}
	あなたの手札 1 枚を廃棄する。それが……
	\begin{itemize}
		\item コスト\Coin{3}以上の場合、他のプレイヤーは全員、
			呪いを獲得する。
		\item 財宝カードの場合、他のプレイヤーは全員、手札が
			3 枚になるように捨札をする。
	\end{itemize}
}[2017私家]
\CardInfo{Rock}{石}+{Cost=4,Type=財宝,MC=1,Copy=5}[このカードは投石機/石の山札に由来する]{
	\Vanilla(1){}
	\hmbar
	これを獲得または廃棄したとき、あなたの購入フェイズである場合は
	銀貨 1 枚をあなたのデッキの上に獲得し、それ以外の場合は銀貨
	1 枚を手札に獲得する。
}[2019私家]
\CardInfo{Chariot Race}{戦車競走}{Cost=3}{
	\Vanilla{action=1}
	あなたのデッキの一番上のカードを公開し、手札に加える。左隣
	のプレイヤーは自分のデッキの一番上のカードを公開する。あなた
	のカードのほうがコストが高い場合、\PlusCoin{1}、\PlusVP{1}。
}
\CardInfo{Enchantress}{女魔術師}{Cost=3,Type=アクション—アタック—持続,Color=持続}{
	あなたの次のターンまで、他のプレイヤーが自分のターン最初
	のアクションカードを使用するとき、そのプレイヤーはその
	指示の代わりに、\PlusCard{1}、\PlusAct{1}を得る。あなたの
	次のターンの開始時に\PlusCard{2}。
}
\CardInfo{Farmers' Market}{農家の市場}{Cost=3,Type=アクション—集合}{
	\Vanilla{buy=1}
	農家の市場の山札に\VP[4]以上ある場合、\VP をすべて得て
	これを廃棄する。そうでない場合、農家の市場の山札に\VP[1]を
	追加し、その後、農家の市場の山札に置かれている\VP[1]につき
	\PlusCoin{1}。
}[2019私家]
\CardInfo{Gladiator/Fortune}{剣闘士/大金}{Cost=2}{
	\hmafont
	この山札は上半分に剣闘士を 5 枚、下半分に大金
	を 5 枚置いてゲームを始める。この山札の一番
	上にあるカードだけが、獲得・購入可能。
}[2017HJ]
\CardInfo{Gladiator}{剣闘士}{Cost=3,Copy=5}[このカードは剣闘士/大金の山札に由来する]{
	\Vanilla{coin=2}
	あなたの手札 1 枚を公開する。左隣のプレイヤーは同じカード
	を自分の手札から公開してもよい。公開しなかった場合、\PlusCoin{1}、
	そしてサプライの剣闘士 1 枚を廃棄する。
}
\CardInfo{Fortune}{大金}{CostA=\Coin{8}\Debt{8},Type=財宝,MC=\(\times\text{2}\),Copy=5}
[このカードは剣闘士/大金の山札に由来する]{
	\Vanilla{buy=1}
	これがこのターンに使用した最初の大金である場合、
	あなたの\Coin{}を倍にする
	(現在の所持\Coin{}と同じ\Coin{}を得る)。
	\hmbar
	これを獲得するとき、あなたの場の剣闘士 1 枚につき
	金貨 1 枚を獲得する。
}[2017私家]
\CardInfo{Sacrifice}{生贄}{Cost=4}{
	あなたの手札 1 枚を廃棄する。そのカードが……
	\begin{itemize}
		\item アクションカードの場合、\PlusCard{2}、
			\PlusAct{2}。
		\item 財宝カードの場合、\PlusCoin{2}。
		\item 勝利点カードの場合、\PlusVP{2}。
	\end{itemize}
}
\CardInfo{Temple}{神殿}{Cost=4,Type=アクション—集合}{
	\Vanilla{VP=1}
	あなたの手札から 1 〜 3 枚の名前の異なるカード
	を廃棄する。神殿の山札の上に\VP[1]を追加する。
	\hmbar
	これを獲得するとき、神殿の山札の上の\VP をすべて
	得る。
}
\CardInfo{Villa}{ヴィラ}{Cost=4}{
	\Vanilla{action=2,buy=1,coin=1}
	\hmbar
	これを獲得するとき、これを手札に加え、\PlusAct{1}。
	あなたの購入フェイズ中の場合、アクションフェイズに戻る。
}
\CardInfo{Archive}{資料庫}{Cost=5,Type=アクション—持続,Color=持続}{
	\Vanilla{action=1}
	あなたのデッキの上から 3 枚を伏せて脇に置く
	(あなたはこれらのカードを見てもよい)。
	現在と、あなたの次の 2 ターンの開始時に、
	その中の 1 枚をあなたの手札に加える。
}
\CardInfo{Capital}{元手}{Cost=5,Type=財宝,MC=6}{
	\Vanilla(6){buy=1}
	\hmbar
	これをあなたの場から捨札にするとき、\Debt{6}を得る。
	その後\Debt{}の返済をしてもよい。
}
\CardInfo{Charm}{御守り}{Cost=5,Type=財宝,MC=?}{
	次のうちの 1 つを選ぶ。
	\begin{itemize}
		\item \PlusCoin{2}、\PlusBuy{1}。
		\item このターン、次にカード 1 枚を購入するとき、
			それと同じコストで名前の異なるカード 1 枚を
			獲得してもよい。
	\end{itemize}
}
\CardInfo{Crown}{冠}+{Cost=5,Type=アクション—財宝,Color=アクション,ColorB=財宝}{
	あなたのアクションフェイズ中の場合、あなたの
	手札のアクションカード 1 枚を 2 回使用してもよい。
	あなたの購入フェイズ中の場合、あなたの手札の
	財宝カード 1 枚を 2 回使用してもよい。
}
\CardInfo{Forum}{公共広場}{Cost=5}{
	\Vanilla{card=3,action=1}
	あなたの手札を 2 枚捨札にする。
	\hmbar
	これを購入するとき、\PlusBuy{1}。
}
\CardInfo{Groundskeeper}{庭師}{Cost=5}{
	\Vanilla{card=1,action=1}
	\hmbar
	これがあなたの場に出ている限り、あなたが勝利点
	カード 1 枚を獲得するとき、\PlusVP{1}。
}
\CardInfo{Legionary}{軍団兵}{Cost=5,Type=アクション—アタック}{
	\Vanilla{coin=3}
	あなたの手札から金貨 1 枚を公開してもよい。
	そうした場合、他のプレイヤーは全員、手札が 2 枚
	になるように捨札にし、その後カードを 1 枚引く。
}
\CardInfo{Wild Hunt}{ワイルドハント}{Cost=5,Type=アクション—集合}{
	次のうち 1 つを選ぶ。
	\begin{itemize}
		\item \PlusCard{3}。ワイルドハントの山札の上に
			\VP[1]を追加する。
		\item 屋敷を獲得する。獲得した場合、ワイルドハント
			の山札の上の\VP をすべて得る。
	\end{itemize}
}
\end{document}

% +++
% latex="texfot lualatex"
% +++
\documentclass{waku}
\begin{document}
\title{ドミニオン整理用ランダマイザ}
\date{\today}
\author{ひとみさん}
\maketitle
\section{使い方}
\begin{enumerate}
	\item 100\% の倍率で印刷します。
	\item 四隅の裁ち切り線の内側で裁断します。
	\item TCG サイズのスリーブに台紙とともに入れます。
\end{enumerate}
\section{特徴}
\begin{itemize}
	\item 追加で必要になるカードが横に示されています。
	\item 最新のルールに則ったテキストを表記しています。
	\item サプライに置かないカードには\ltjjachar`★の印を
		つけています。
	\item それだけの山札がないカードには\ltjjachar`☆の印を
		つけています。
\end{itemize}
\pagebreak
\section{注意}
\begin{itemize}
	\item 日本語テキストは、ドミニオンオンラインに準拠
		しています。
	\item ドミニオンオンラインのテキストがわかりにくい
		場合、物理版の日本語版を参照したり、作者が翻訳
		を作成したりした場合があります。
	\item 参照した日本語訳が作成された年度は右下に記載
		されています。HJ と記載されている場合、物理版の
		日本語版準拠のテキスト、私家とある場合、作者
		が作成した翻訳になります。
	\item ウィル・オ・ウィスプ(夜想曲)は「鬼火」と
		表記している部分があります。
\end{itemize}
\section{翻訳の方針}
\begin{itemize}
	\item deck の訳語は(山札ではなく)デッキとする。
	\item pile の訳語は山札とする。
	\item costing up to \Coin{} は(安いではなく)
		〜よりコストが低いと訳する。
	\item take(横向きのカードを手元に移すこと)は
		(得るではなく)取ると訳する。
	\item take (トークンを得ること)は得ると訳する。
	\item 手札、デッキ、場に対する所有代名詞は訳出する。
\end{itemize}
\end{document}

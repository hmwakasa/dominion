% +++
% latex="texfot lualatex"
% +++
\documentclass[landscape]{waku}
\begin{document}
\def\Expantion{ルネサンス}\def\errata{2019}
\def\Type{アーティファクト}
\LandscapeInfo{Flag}{旗}+(旗手)*{}{
	あなたの手札を引き直すとき、\PlusCard{1}。
}[2020HJ]
\LandscapeInfo{Horn}{角笛}(国境警備隊)*{}{
	1 ターンに 1 回のみ、あなたが場から国境警備隊を
	捨札にするとき、それをデッキの一番上に置いてもよい。
}
\LandscapeInfo{Key}{鍵}+(出納官)*{}{
	あなたのターン開始時に \PlusCoin{1}。
}
\LandscapeInfo{Lantern}{ランタン}+(国境警備隊)*{}{
	あなたの使用する国境警備隊は、カードを 3 枚
	公開し、2 枚を捨札にする。(3 枚ともアクション
	カードの場合は角笛を得る。
}[2019私家]
\LandscapeInfo{Treasure Chest}{宝箱}(剣客)*{}{
	あなたの購入フェイズの開始時に、金貨 1 枚を獲得する。
}
\def\Type{プロジェクト}
\LandscapeInfo{Cathedral}{大聖堂}{Cost=3}{
	あなたのターンの開始時に、あなたの手札 1 枚を廃棄する。
}
\LandscapeInfo{City Gate}{城門}{Cost=3}{
	あなたのターンの開始時に、\PlusCard{1}。その後
	あなたの手札 1 枚をデッキの一番上に置く。
}
\LandscapeInfo{Pageant}{野外劇}{Cost=3}{
	あなたの購入フェイズの終了時に、\Coin{1} を払って
	\PlusCoffer{1} を得てもよい。
}
\LandscapeInfo{Sewers}{下水道}{Cost=3}{
	あなたがこの効果以外でカード 1 枚を廃棄するとき、
	追加であなたの手札 1 枚を廃棄してもよい。
}
\LandscapeInfo{Star Chart}{星図}{Cost=3}{
	あなたがデッキをシャッフルするとき、その中の 1 枚を
	選んでその一番上に置いてもよい。(シャッフルする
	カードをすべて見て一番上に置くカードを決めて、その
	時にシャッフルされない残りのデッキのカードを見てもよい)
}[2021私家]
\LandscapeInfo{Exploration}{探査}{Cost=4}{
	あなたの購入フェイズの終了時、その購入フェイズで
	何もカードを購入していない場合、\PlusCoffer{1}、
	\PlusVillager{1}。
}[2021私家]
\LandscapeInfo{Fair}{縁日}{Cost=4}{
	あなたの購入フェイズの開始時、\PlusBuy{1}。
}
\LandscapeInfo{Silos}{サイロ}{Cost=4}{
	あなたのターンの開始時に、好きな枚数の銅貨を公開して
	捨札にし、同じ枚数のカードを引く。
}
\LandscapeInfo{Sinister Plot}{悪巧み}{Cost=4}{
	あなたのターンの開始時に、ここにあなたのトークンを 1 個
	加える、または、ここにあるあなたのトークンすべてを取り除き、
	取り除いたトークン 1 個につき\PlusCard{1}。
}[2021私家]
\LandscapeInfo{Academy}{学園}{Cost=5}{
	あなたがアクションカード 1 枚を獲得するとき、\PlusVillager{1}。
}
\LandscapeInfo{Capitalism}{資本主義}{Cost=5}{
	あなたのターン中、指示文に「\PlusCoin{}」と書いてある
	アクションカードは、財宝カードでもある。
}
\LandscapeInfo{Fleet}{艦隊}{Cost=5}{
	ゲーム終了時、これを持つプレイヤーのみ、追加で 1ターン
	を得る。
}
\LandscapeInfo{Guildhall}{ギルド集会所}{Cost=5}{
	あなたが財宝カード 1 枚を獲得するとき、\PlusCoffer{1}。
}
\LandscapeInfo{Piazza}{ピアッツァ}{Cost=5}{
	あなたのターンの開始時に、デッキの一番上のカード
	を公開する。それがアクションカードの場合、それを
	使用する。
}
\LandscapeInfo{Road Network}{道路網}{Cost=5}{
	他のプレイヤーが勝利点カード 1 枚を獲得するとき、
	\PlusCard{1}。
}
\LandscapeInfo{Barracks}{兵舎}{Cost=6}{
	あなたのターンの開始時に、\PlusAct{1}。
}
\LandscapeInfo{Crop Rotation}{輪作}{Cost=6}{
	あなたのターンの開始時に、勝利点カード 1 枚を捨札に
	してもよい。そうした場合、\PlusCard{2}。
}
\LandscapeInfo{Innovation}{技術革新}{Cost=6}{
	あなたのターン中、最初にアクションカードを獲得する
	とき、そのカードを使用してもよい。(カードを場に
	出せない限り使用できない)
}[2021私家]
\LandscapeInfo{Canal}{運河}{Cost=7}{
	あなたのターン中、カードのコストは\Coin{1}下がる。
}
\LandscapeInfo{Citadel}{山砦}{Cost=8}{
	あなたのターン中、最初にアクションカードを使用する
	ときは、そのカードを 1 回ではなく 2 回使用する。
}[2021私家]
\end{document}

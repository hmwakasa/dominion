% +++
% latex="texfot lualatex"
% +++
\documentclass{waku}
\begin{document}
\def\Expantion{ルネサンス}\def\errata{2019}
\CardInfo{Border Gurd}{国境警備隊}[ランタン, 角笛]{Cost=2,}{
	\Vanilla{action=1}
	あなたのデッキの上から 2 枚のカードを公開する。
	そのうち片方をあなたの手札に加え、残りを捨て札にする。
	両方がアクションカードの場合、ランタンか角笛を取る。
}[2020HJ]
\CardInfo{Ducat}{ドゥカート金貨}{Cost=2,Type=財宝}{
	\Vanilla{coffer=1,buy=1}
	\hmbar
	これを獲得するとき、手札の銅貨 1 枚を廃棄してもよい。
}
\CardInfo{Lackeys}{追従者}{Cost=2,}{
	\Vanilla{card=2}
	\hmbar
	これを獲得するとき、\PlusVillager{2}。
}
\CardInfo{Acting Troupe}{劇団}{Cost=3,}{
	\Vanilla{villager=4}
	これを廃棄する。
}
\CardInfo{Cargo Ship}{貨物船}{Cost=3,Type=アクション—持続,Color=持続}{
	\Vanilla{coin=2}
	あなたがこのターンに獲得するカードのうち 1 枚を、
	(このカードの)脇に表向きに置いてもよい。あなたの
	次のターンの開始時に、それをあなたの手札に加える。
}
\CardInfo{Experiment}{実験}{Cost=3,}{
	\Vanilla{card=2,action=1}
	これをサプライに戻す。
	\hmbar
	これを獲得するとき、追加で実験 1 枚を獲得する(その
	実験は追加の実験を獲得させない)。
}
\CardInfo{Improve}{増築}{Cost=3,}{
	\Vanilla{coin=2}
	クリーンアップフェイズの開始時に、あなたはこの
	ターンに場から捨て札にするアクションカード 1 枚を
	廃棄してもよい。そうした場合、それよりコスト
	が\Coin{1}高いカード 1 枚を獲得する。
}
\CardInfo{Flag Bearer}{旗手}[旗]{Cost=4,}{
	\Vanilla{coin=2}
	\hmbar
	これを獲得または廃棄するとき、旗を取る。
}
\CardInfo{Hideout}{根城}{Cost=4,}{
	\Vanilla{card=1,action=2}
	あなたの手札 1 枚を廃棄する。それが勝利点カードの場合、
	呪い 1 枚を獲得する。
}
\CardInfo{Inventor}{発明家}{Cost=4,}{
	コスト\Coin{4}以下のカード 1 枚を獲得する。この
	ターン、(サプライ以外も含めた)すべてのカードの
	コストは\Coin{1}下がる。
}
\CardInfo{Mountain Village}{山村}{Cost=4,}{
	\Vanilla{action=2}
	あなたの捨札をすべて見て、その中の 1 枚をあなたの手札
	に加える。手札に加えられない場合、\PlusCard{1}。
}
\CardInfo{Patron}{パトロン}{Cost=4,Type=アクション—リアクション,Color=リアクション,}{
	\Vanilla{villager=1,coin=2}
	\hmbar
	アクションフェイズ中、これが公開されるとき、\PlusCoffer{1}
	(「公開する」という語が用いられている必要がある)。
}[2021私家]
\CardInfo{Priest}{司祭}{Cost=4,}{
	\Vanilla{coin=2}
	あなたの手札 1 枚を廃棄する。その後、このターンの間、
	あなたがカードを 1 枚を廃棄するとき、\PlusCoin{2}。
}[2020HJ]
\CardInfo{Research}{研究}{Cost=4,Type=アクション—持続,Color=持続}{
	\Vanilla{action=1}
	あなたの手札 1 枚を廃棄する。そのコスト\Coin{1}につき 1 枚、
	あなたのデッキの上のカードを(このカードの)脇に伏せて置く。
	あなたの次のターンの開始時に、それらをあなたの手札に加える。
}
\CardInfo{Silk Merchant}{絹商人}{Cost=4,}{
	\Vanilla{card=2,buy=1}
	\hmbar
	これを獲得または廃棄するとき、\PlusVillager{1}、
	\PlusCoffer{1}。
}
\CardInfo{Old Witch}{老魔女}{Cost=5,Type=アクション—アタック}{
	\Vanilla{card=3}
	他のプレイヤーは全員、呪い 1 枚を獲得する。その後、
	他のプレイヤーは全員、手札の呪い 1 枚を廃棄してもよい。
}
\CardInfo{Recruiter}{徴募官}{Cost=5,}{
	\Vanilla{card=2}
	あなたの手札 1 枚を廃棄する。そのコスト\Coin{1}につき
	\PlusVillager{1}。
}
\CardInfo{Scepter}{王笏}{Cost=5,Type=財宝,MC=?}{
	次のうち 1 つを選ぶ:
	\begin{itemize}
		\item \PlusCoin{2}
		\item このターンにあなたが使用した、場に残っている
			アクションカード 1 枚を再使用する。
	\end{itemize}
}
\CardInfo{Scholar}{学者}{Cost=5,}{
	あなたの手札をすべて捨てる。
	\Vanilla{card=7}
}
\CardInfo{Sculptor}{彫刻家}{Cost=5}{
	コスト\Coin{4}以下のカード 1 枚をあなたの手札に獲得する。
	それが財宝カードの場合、\PlusVillager{1}。
}
\CardInfo{Seer}{先見者}{Cost=5}{
	\Vanilla{card=1,action=1}
	あなたのデッキの上から 3 枚を公開する。その中から
	コスト\Coin{2}から\Coin{4}のカードをあなたの手札に加える。
	残りを好きな順番でデッキに戻す。
}
\CardInfo{Spices}{香辛料}{Cost=5,Type=財宝,MC=2}{
	\Vanilla(2){buy=1}
	\hmbar
	これを獲得するとき、\PlusCoffer{2}。
}
\CardInfo{Swashbuckler}{剣客}[宝箱]{Cost=5}{
	\Vanilla{card=3}
	あなたの捨札にカードがある場合:
	\begin{quote}
		\PlusCoffer{1}、4 枚以上財源トークンを所持
		している場合、宝箱を取る。
	\end{quote}
}
\CardInfo{Treasurer}{出納官}[鍵]{Cost=5}{
	\Vanilla{coin=3}
	次のうち 1 つを選ぶ:
	\begin{itemize}
		\item あなたの手札の財宝カード 1 枚を廃棄する。
		\item 廃棄置き場の財宝カード 1 枚をあなたの手札に獲得する。
		\item 鍵を取る。
	\end{itemize}
}
\CardInfo{Villain}{悪党}{Cost=5,Type=アクション—アタック}{
	\Vanilla{coffer=2}
	手札が 5 枚以上の他のプレイヤーは全員、
	コスト\Coin{2}以上のカード 1 枚を捨札にする
	(ない場合、手札を公開する)。
}
\end{document}

% +++
% latex="texfot lualatex"
% +++
\documentclass{waku}
\begin{document}
\def\Expantion{移動動物園}\def\errata{2020}
\CardInfo{Black Cat}{黒猫}{Cost=2,Type=アクション—アタック—リアクション,Color=リアクション}{
	\Vanilla{card=2}
	あなたのターンでない場合、他のプレイヤーは全員、
	呪い 1 枚を獲得する。
	\hmbar
	他のプレイヤーが勝利点カードを獲得するとき、
	あなたは手札からこれを使用してもよい。
}
\CardInfo{Sleigh}{そり}[馬]{Cost=2,Type=アクション—リアクション,Color=リアクション}{
	馬 2 枚を獲得する。
	\hmbar
	あなたがカード 1 枚を獲得するとき、それをあなたの手札に
	加えるかあなたのデッキの上に置くために、これを捨札にして
	よい。
}
\CardInfo{Supply}{配給品}[馬]{Cost=2,Type=財宝,MC=1}{
	\Vanilla(1){}
	馬 1 枚をあなたのデッキの上に獲得する。
}
\CardInfo{Camel Train}{ラクダの隊列}{Cost=3,}{
	サプライから勝利点以外のカードを 1 枚追放する。
	\hmbar
	これを獲得するとき、サプライの金貨 1 枚を追放する。
}
\CardInfo{Goatherd}{ヤギ飼い}{Cost=3,}{
	\Vanilla{action=1}
	あなたの手札 1 枚を廃棄してもよい。右隣のプレイヤーの
	最後のターンに、そのプレイヤーが廃棄したカード 1 枚に
	つき\PlusCard{1}。
}
\CardInfo{Scrap}{がらくた}[馬]{Cost=3,}{
	あなたの手札 1 枚を廃棄する。そのコスト\Coin{1}につき、
	次から異なるものを次の中から選び、上から順に実行する。
	\begin{itemize}
		\item \PlusCard{1}
		\item \PlusAct{1}
		\item \PlusBuy{1}
		\item \PlusCoin{1}
		\item 銀貨 1 枚を獲得する。
		\item 馬 1 枚を獲得する。
	\end{itemize}
}[2020私家]
\CardInfo{Sheepdog}{牧羊犬}{Cost=3,Type=アクション—リアクション,Color=リアクション}{
	\Vanilla{card=2}
	\hmbar
	あなたがカード 1 枚を獲得するとき、手札から
	これを使用してもよい。
}
\CardInfo{Snowy Village}{雪深い村}{Cost=3}{
	\Vanilla{card=1,action=4,buy=1}
	このターン、今後得る\PlusAct{}をすべて無視する。
}
\CardInfo{Stockpile}{備蓄品}{Cost=3,Type=財宝,MC=3}{
	\Vanilla(3){buy=1}
	これを追放する。
}
\CardInfo{Bounty Hunter}{賞金稼ぎ}{Cost=4}{
	\Vanilla{action=1}
	あなたの手札 1 枚を追放する。あなたの追放の中にそのカードと
	同じカードがなかった場合、\PlusCoin{3}。
}[2020HJ]
\CardInfo{Cardinal}{枢機卿}{Cost=4,Type=アクション—アタック}{
	\Vanilla{coin=2}
	他のプレイヤーは全員、デッキの上から 2 枚を公開し、
	その中のコストが\Coin{3}から\Coin{6}のカード 1 枚
	を追放し、残りを捨札にする。
}
\CardInfo{Cavalry}{騎兵隊}[馬]{Cost=4,}{
	馬 2 枚を獲得する。
	\hmbar
	これを獲得するとき、\PlusCard{2}、\PlusBuy{1}、
	あなたの購入フェイズ中の場合、アクションフェイズ
	に戻る。
}
\CardInfo{Groom}{馬丁}[馬]{Cost=4}{
	コスト\Coin{4}以下のカード 1 枚を獲得する。
	獲得したカードが……
	\begin{itemize}
		\item アクションカードの場合、馬 1 枚を獲得する。
		\item 財宝カードの場合、銀貨 1 枚を獲得する。
		\item 勝利点カードの場合、\PlusCard{1}、\PlusAct{1}。
	\end{itemize}
	(複数該当する場合、順に実行する)
}
\CardInfo{Hostelry}{旅籠}[馬]{Cost=4}{
	\Vanilla{card=1,action=2}
	\hmbar
	これを獲得するとき、好きな枚数の財宝カードを
	公開して捨札にしてもよい。そうした場合、同じ
	枚数の馬を獲得する。
}
\CardInfo{Village Green}{村有緑地}{Cost=4,Type=アクション—持続—リアクション,Color=持続,ColorB=リアクション}{
	現在またはあなたの次のターンの開始時に、
	\Vanilla{card=1,action=2}
	\hmbar
	このカードをクリーンアップフェイズ以外の場面で
	捨札にするとき、これを使用してもよい。
}[2020私家]
\CardInfo{Barge}{艀}+{Cost=5,Type=アクション—持続,Color=持続}{
	現在またはあなたの次のターンの開始時に、
	\Vanilla{card=3,buy=1}
}
\CardInfo{Coven}{魔女の集会}{Cost=5,Type=アクション—アタック}{
	\Vanilla{action=1,coin=2}
	他のプレイヤーは全員、サプライの呪い 1 枚を追放する。
	追放できなかった場合、そのプレイヤーは追放している
	呪いをすべて捨札にする。
}
\CardInfo{Displace}{強制退去}{Cost=5}{
	あなたの手札 1 枚を追放する。それよりコストが最大\Coin{2}高い
	名前の異なるカード 1 枚を獲得する。
}
\CardInfo{Falconer}{鷹匠}{Cost=5,Type=アクション—リアクション,Color=リアクション}{
	これよりコストが少ないカード 1 枚をあなたの手札に獲得する。
	\hmbar
	誰かが種類(アクション、アタック等)を 2 つ以上持つ
	カード 1 枚を獲得するとき、手札からこれを使用してもよい。
}
\CardInfo{Gatekeeper}{門番}{Cost=5,Type=アクション—持続—アタック,Color=持続}{
	あなたの次のターン開始時に\PlusCoin{3}。それまでの間、
	他のプレイヤーが自分の追放にないアクションカードまたは
	財宝カード 1 枚を獲得するとき、そのプレイヤーはそれを
	追放する。
}
\CardInfo{Hunting Lodge}{狩猟小屋}{Cost=5}{
	\Vanilla{card=1,action=2}
	あなたの手札をすべて捨札にしてもよい。そうした場合\PlusCard{5}。
}
\CardInfo{Kiln}{炉}+{Cost=5}{
	\Vanilla{coin=2}
	このターンあなたが次にカード 1 枚を使用するとき、先に
	そのカードと同じカード 1 枚を獲得してもよい。
}
\CardInfo{Livery}{貸し馬屋}[馬]{Cost=5}{
	\Vanilla{coin=3}
	このターン、コスト\Coin{4}以上のカードを獲得するとき、
	馬 1 枚を獲得する。
}
\CardInfo{Mastermind}{首謀者}{Cost=5,Type=アクション—持続,Color=持続}{
	あなたの次のターンの開始時に、手札のアクションカード 1 枚
	を 3 回使用してもよい。
}[2020HJ]
\CardInfo{Paddock}{パドック}[馬]{Cost=5}{
	\Vanilla{coin=2}
	馬 2 枚を獲得する。サプライの空の山札 1 つにつき
	\PlusAct{1}。
}
\CardInfo{Sanctuary}{聖域}{Cost=5}{
	\Vanilla{card=1,action=2,buy=1}
	あなたの手札 1 枚を追放してもよい。
}
\CardInfo{Fisherman}{漁師}{CostA=\Coin{5}*,}{
	\Vanilla{card=1,action=1,coin=1}
	\hmbar
	ターン中、ターンプレイヤーの捨札置き場にカードが 1 枚
	もない場合、このカードのコストは\Coin{3}少なく
	なる。
}
\CardInfo{Destrier}{デストリエ}{CostA=\Coin{6}*}{
	\Vanilla{card=2,action=1}
	\hmbar
	ターン中、このカードのコストはそのターンにターンプレイヤー
	が獲得したカード 1 枚につき\Coin{1}下がる。
}
\CardInfo{Wayfarer}{行人}{CostA=\Coin{6}*}{
	\Vanilla{card=3}
	銀貨 1 枚を獲得してもよい。
	\hmbar
	このカードのコストは、このターン誰かがカードを獲得して
	いた場合、そのうち最後に獲得されたカード他のカードと
	同じになる。
}
\CardInfo{Animal Fair}{動物見本市}{CostA=\Coin{7}*}{
	\Vanilla{coin=3}
	サプライの空の山札 1 つにつき、\PlusBuy{1}。
	\hmbar
	このカードのコストを支払う代わりに、あなたは手札のアクション
	カード 1 枚を廃棄してもよい。
}
\CardInfo{Horse}{馬}+*{CostA=\Coin{3}*,Copy=30}{
	\Vanilla{card=2,action=1}
	これを馬の山札に戻す。
}
\end{document}
